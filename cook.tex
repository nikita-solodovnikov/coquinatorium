\documentclass[11pt,a5paper]{article}
\usepackage[utf8]{inputenc}
\usepackage[russian]{babel}
\usepackage{geometry}

\begin{document}
\section{Закуски}
\paragraph{Маринованный дайкон} % Платоник

Дайкон, шафран, рисовый уксус, сахар, соль.

Запихнуть в банку дайкон, нарезанный в мужской мизинец толщиной.
Заварить несколько тычинок шафрана кипятком.
В рисовом уксусе растворить две ложки сахару и две ложечки соли (на 200 грамм уксуса).

Залить банку процеженным настоем и долить уксусом. Можно добавить чуть рисового вина и/ли соевого соуса.

Ждать неделю.

Хорошо к выпивке и дальневосточной кухне.

\paragraph{Квашеная свекла}

Свекла, соль.

Свеколки нарезать дольками, упихать в банку, залить рассолом (на литр воды 2-3 десертные ложки соли, не больше!). Оставить кваситься при комнатной температуре на неделю, снимая иногда плесень. Должно подниматься пена. Свекла станет мягкой, рассол --- кисловатым. 

Подавать как закуску или дополнительный гарнир.

\paragraph{Малосольные огурцы на скорую руку} % Платоник

Огурцы, чеснок, укроп, хрен, соль.

На двухлитровую банку возьмите килограмм огурцов, помойте, обрежьте кончики и укладывайте плотно в банку, перемежая слои зубчиками чеснока (головка или две) и укропом (можно взять пучок на всё). Сверху положите три ложки хрена и залейте кипящим рассолом -- 3 десертные ложки соли на литр воды.

На двухлитровую банку у меня получился 1 килограммм огурцов и ровно 1 литр рассола. Мне не показалось вкусным, готовил один раз.

\paragraph{Carrottes marinèes} % Платоник

Морковь, чеснок, укроп, петрушка, горчица, белое вино, винный уксус, прованское масло, соль, сахар.

Нарежьте морковь пластинами с указательный палец, на килограмм моркови в кастрюле смешайте стакан воды, винного уксуса и белого вина, растворите в этой смеси по две ложечки соли и сахару, две щепоти красного острого перца, добавьте стакан прованского масла и поставьте на сильный огонь. Как вскипит, бросьте два разрезанных пополам и раздавленных ножом зубка чесноку, по паре веточек укропа и петрушки, опустите туда морковь на 3 минуты, а затем выложите в банку и как маринад остынет, размешайте в нём две ложки горчицы -- и залейте морковь.

Ждать 3 дня.

\section{Яйца}

\paragraph{Французский омлет} % Платоник
4 яйца взбить, добавить соль и черный перец. Можно добавить зелень и тертый (не мягкий) сыр. На разогретую (не раскаленную) и смазанную сливочным маслои сковороду вылить смесь. Пододвигать с краев лопаткой, чтобы жидкая часть стекала на освободившееся место. Как нечему будет стекать (пусть верх еще и <<сопливый>>), готово.

Нужен навык --- пододвигать и наклонять сковороду, чтобы перетекало.

\paragraph{Omelette lorraine} % Платоник
Взбейте 5 яиц, 4 ложки натёртого не мягкого сыра, ложку сливок (молока; воды). На сковороде обжарьте до хруста тонкие ломтики грудинки и залейте смесью, жарьте как обычный французский омлет.

\paragraph{Молдавский омлет} % Платоник
Взбейте 4 яйца, полстакана зернистой брынзы, полстакана молока, две ложки мелко порубленной зелени, от души красного перцу. Мелко порубить 4 зубка чесноку. Разогрейте постное масло, ароматизируйте чесноком, залейте смесь, жарьте как обычный омлет.

\paragraph{Чирбули красное} % Платоник
4 яйца, 2-3 томата, лук, чеснок, свежий базилик, ткемали, мука, сливочное или топленое масло.

Томаты натрите на тёрке, половину головки чесноку и 2 луковицы мелко нарубите. Растопите масло в глубокой сковороде, ароматизируйте его чесноком и затем спассеруйте лук. Отправьте туда же помидоры (они кислые; лучше использовать не чугун), дайте закипеть, добавьте ткемали, дайте закипеть, просейте туда же ложку муки и перемешайте, как возникнет опасность, что подгорит, влейте стакан кипятку, пусть немного побулькает. Скатывайте яйца по одному так, чтобы ни желток, ни белок не растекались. Подождите, пока белок загустеет, посыпьте базиликом.

Подавайте с хлебом навроде лепешки, чтобы вымакивать жижу.



\section{Постные супы}
\paragraph{Армянский чечевичный (воспнапур)}
\paragraph{Суп из зеленой фасоли}
\paragraph{Шечаманды}
\paragraph{Чечевичная похлебка}
\section{Постные вторые блюда}

\subsection{Овощи}

\paragraph{Жареные овощи по-молдавски} % Платоник
Кабачок/ баклажаны/ чушка, соль, перец, постное масло, (винный) уксус.

Овощи нарезать шайбами/ чушку -- вдоль. Обжаривать в воке в постном масле до золотистого цвета. Укладывать в миску плотными слоями (порядок не важен), посыпая каждый слой солью, перцем и сбрызгивая уксусом. Поставить под гнёт хотя бы на 20 минут, чтобы пропиталось смесью.

\paragraph{Картофельная яхния} % Платоник

Картофель, морковь, лук, чушка, помидоры, постное масло, черный перец горошком, красный перец, чабрец/ сушеная мята/ смесь прованских трав.

Две луковицы нарежьте небольшими кубиками, две чушки так же, две моркови тонкими кружками или полукружьями, килограмм картофеля --- половинками или четвертинками. Молодой можно не чистить. 3-4 помидора натереть на терке.

В казане спассеровать лук до прозрачности, добавить чушку и пассеровать, пока не станет мягкой. Отправить к зажарке картофель и морковь, помешать, влить стакан кипятку и тушить под крышкой на небольшом огне до готовности картофеля. Влить туда же помидоры, положить 5-6 горошин перца, добавить от души красного перца. Можно загустить мукой и смягчить сахаром. Добавить на выбор: чабрец (самый <<болгарский>> вкус), сушеную мяту или смесь прованских трав -- и потушить еще 7-10 минут. Обильно посыпать петрушкой, перемешать, снять с огня.

\paragraph{Имам-баялды}

\paragraph{Аджапсандал}

\subsection{Крупы, бобовые}

\paragraph{...}

Замочите часов на 10 горох, фасоль или чечевицу, на медленном огне отварите с кореньями (коренья затем выньте), подсолите под конец и вмешайте пахучего постного масла, всыпьте нарубленные чеснок и зелень.


\paragraph{Фасолица}

Отвари фасоль, пока не остыла, протри через сито и вмешивай подогретое постное масло и растертый с солью чеснок и/ли лук (жареный или сырой). Должно получиться что-то вроде хумуса.

\paragraph{Кашица тихвинская} % Платоник

Горох, гречневый продел, лук, постное масло.

Горсть гороха замочите с вечера, разварите и досыпьте две горсти гречневого продела (но и цельная подойдет), а пока варится, спассеруйте до золотистости две мелко порезанных луковицы, налив в сковороду побольше масла, чтобы получился лук с маслом. Вмешайте его в кашицу.

\paragraph{Полба с сельдереем} % eda.ru, Полба с кореньями

Полба, корень сельдерея или иные коренья, постное масло.

Отварить две части полбы (цельной) в четырех части воды, а рядом пассеровать нарезанные мелкими кубиками коренья (одну часть). Вмешать.


\subsection{Черепокожные}
\paragraph{Кальмары фри} % Платоник

Тушки кальмаров, перец, красный перец, кардамон, кориандр, соевый соус, постное масло.

Кальмары выпотрошить и очистить от пленки, разрезать на три части (оперение и две половины), чтобы получились плоские части. Омыть, провести надрезы сеточкой шагом в сантиметр или меньше. Засыпать специями: немного жгучего красного перца, прилично черного, немного кориандра и побольше кардамона; влить 2-3 ложки соевого соуса на полкило кальмаров. Подождать, чтобы специи разошлись-впитались.

В воке на сильном огне (и его далее не сбавлять) разогреть масло до дымка, отправить в него треть кальмаров --- дадут сок -- приподнять их и подождать, пока сок выкипит (осторожно, масло будет плеваться). Как отплюётся, опустить кальмаров обратно в масло, жарить (важно: не тушить) две минуты. Повторить с остальными частями.

На вкус должно напоминать о гриле;

на 700 граммов тушек у меня получилось 500 граммов чистого сырого кальмара --- и 300 грамм после поджаривания.
\paragraph{Креветки индийским манером} % Платоник

Очищенные креветки, лучше крупные, две салатных луковицы, лучше белые, но и красные подойдут, томаты (<<в собственном соку>> или свежие), чеснок, имбирь, паприка (лучше копченая), лютые чушки (лучше сушеные), кинза.

Креветки заспьте ложкой карри, потрясите и дайте полчаса-час настояться в холодильнике. 

В это время порежьте короткой соломкой два-три зубка чеснока и кусочек имбирного корня, лук --- полукольцами, кинзу как угодно. В казане (можно и в воке) разогрейте масло (топленое, но можно и прованское для жарки), отправьте туда чеснок с имбирем, чуть позже засыпьте луком. Минуты через 3 положите в зажарку 3 стручка сушеной лютой чушки, столовую ложку карри, чайную --- паприки. Еще минуту.

Добавьте банку помидоров (можно натереть на терке). Перемешайте, еще минуту. Добавьте креветки, минуту, кинзу, накройте крышкой и дайте настояться пять минут, выключив огонь.

Подавать с холодным пивом и лепешкой.

\subsection{Макароны}
\paragraph{Spaghetti aglio, olio e peperoncino} % Платоник

Спагетти, чеснок, лютая чушка, оливковое масло.

В сковороду налить оливкового масла и тонко нарезать от половины до одной головки чеснока и одну-полторы лютые чушки. Греть на слабом огне, чтобы не обжаривалось, а давало вкус. Отварить аль денте хорошие спагетти. Смешать.

\paragraph{Удон к креветками и овощами} % Платоник

Китайская лапша --- лучше широкая пшеничная, очищенные креветки, морковь, 2-3 чушки разного цвета, лютая чушка, имбирь, чеснок, зеленый лук, прованское масло, кунжутное масло, соевый соус, рыбный или устричный соус, сахар.

Вок, казан хороши.

Мелко порубите 3 сантиметра корня имбиря и 3-4 зубка чесноку, лютую чушку порежьте тоненько наискосок. Сладкие чушки --- тонкой соломкой (можно брать по половине чушки или меньше), одну морковь тонкими ломтиками наискось, а затем соломкой. Зеленый лук --- наискосок с интервалом в два пальца.

Отварите граммов 100 очищенных креветок, откиньте, в той же воде отварите удон (смотрите, чтобы не слипся).

Разогрейте посильнее прованское масло в воке, высыпьте туда имбирь, чеснок и лютую чушку; через полминуты добавьте креветки; через две минуты --- две столовые ложки сахара; через полминуты, как запахнет карамелью, по столовой ложке соевого и рыбного (устричного) соуса; с через минуту-полторы, как выпарится, добавьте чушки и морковь; как овощи станут мягкими, добавьте лапшу с кунжутным маслом. Попробовать, добавить зелень, готово.



\end{document}
